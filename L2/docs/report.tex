\documentclass[a4paper, 14pt]{extarticle}
\usepackage{./styles/generalPreamble}
\usepackage{./styles/longReportFormat}
\usepackage{./styles/russianLocale}
\usepackage{./styles/nonFancyTOC}
\usepackage{./styles/sourceCode}
\usepackage{./styles/gostBibTex}

\bibliography{bibliography}

\newcommand{\Addon}[3]{%
    \section*{#1}
    \addonsubheader{#2}
    {\fontsize{12pt}{1.5pt}\selectfont
    \lstinputlisting[language=C++, style=num]{#3}
    }
}

\begin{document}
\begin{titlepage}
    \centering
    {\bfseries
        \uppercase{
            Минобрнауки России \\
            Санкт-Петербургский государственный \\
            Электротехнический университет \\
            \enquote{ЛЭТИ} им. В.И.Ульянова (Ленина)\\
        }
        Кафедра МО ЭВМ

        \vspace{\fill}
        \uppercase{Отчёт} \\
        по лабораторной работе №2 \\
        по дисциплине \enquote{Конструирование ПО} \\
        Тема: Разработка приложений
    }

    \vspace{\fill}
    \begin{tabularx}{0.8\textwidth}{l X c r}
        Студент гр. 6304 & & \underline{\hspace{3cm}} & Корытов П.В.\\
        Преподаватель & & \underline{\hspace{3cm}} & Спицин А.В.
    \end{tabularx}

    \vspace{1cm}
    Санкт-Петербург \\
    \the\year{}
\end{titlepage}

\tableofcontents{}
\newpage

\section{Постановка задачи}
\subsection{Цель работы}
\lipsum[1] %TODO

\subsection{Формулировка задания}
\lipsum[1] %TODO

\subsection{Индивидуальное задание}
\begin{itemize}
    \item Фигуры --- пентаграмма, кусок арктангенса, текст, текст в пентаграмме.
    \item Контейнер --- хэш-таблица на базе списка.
\end{itemize}

\section{Ход работы}
\lipsum[1] %TODO

\section{Выводы}
\lipsum[1] %TODO

\printbibliography{}
\addcontentsline{toc}{section}{Список литературы}

\phantomsection{}\addcontentsline{toc}{section}{Приложения}
\Addon{Приложение А}{Исходный код main.cpp}{../src/main.cpp}
\Addon{Приложение Б}{Исходный код mainwindow.h}{../src/mainwindow.h}
\Addon{Приложение В}{Исходный код mainwindow.cpp}{../src/mainwindow.cpp}
\Addon{Приложение Г}{Исходный код hashMap.h}{../src/hashMap.h}
\Addon{Приложение Е}{Исходный код exception.h}{../src/exception.h}
\Addon{Приложение Ж}{Исходный код figures/shape.h}{../src/figures/shape.h}
\Addon{Приложение З}{Исходный код figures/shape.cpp}{../src/figures/shape.cpp}
\Addon{Приложение И}{Исходный код figures/atansegment.h}{../src/figures/atansegment.h}
\Addon{Приложение К}{Исходный код figures/atansegment.cpp}{../src/figures/atansegment.cpp}
\Addon{Приложение Л}{Исходный код figures/pentagram.h}{../src/figures/pentagram.h}
\Addon{Приложение М}{Исходный код figures/pentagram.cpp}{../src/figures/pentagram.cpp}
\Addon{Приложение Н}{Исходный код figures/text.h}{../src/figures/text.h}
\Addon{Приложение О}{Исходный код figures/text.cpp}{../src/figures/text.cpp}
\Addon{Приложение П}{Исходный код figures/pentagramtext.h}{../src/figures/pentagramtext.h}
\Addon{Приложение Р}{Исходный код figures/pentagramtext.cpp}{../src/figures/pentagramtext.cpp}
\Addon{Приложение С}{Исходный код graphwidget.h}{../src/graphwidget.h}
\Addon{Приложение Т}{Исходный код graphwidget.cpp}{../src/graphwidget.cpp}
\Addon{Приложение У}{Исходный код adddialog.h}{../src/adddialog.h}
\Addon{Приложение Ф}{Исходный код adddialog.cpp}{../src/adddialog.cpp}
\Addon{Приложение Х}{Исходный код point.h}{../src/point.h}
\Addon{Приложение Ц}{Исходный код point.cpp}{../src/point.cpp}

\end{document}
