\documentclass[a4paper, 14pt]{extarticle}
\usepackage{./generalPreamble}
\usepackage{./longReportFormat}
\usepackage{./sourceCode}

\begin{document}
    
\begin{titlepage}
    \centering
    {\bfseries
        \uppercase{
            Минобрнауки России \\
            Санкт-Петербургский государственный \\
            Электротехнический университет \\
            \enquote{ЛЭТИ} им. В.И.Ульянова (Ленина)\\
        }
        Кафедра МО ЭВМ

        \vspace{\fill}
        \uppercase{Лабораторная работа} \\
        по дисциплине \enquote{Конструирование ПО} \\
        Тема: Программирование контейнерных классов
    }

    \vspace{\fill}
    \begin{tabularx}{0.8\textwidth}{l X c r}
        Студент гр. 6304 & & \underline{\hspace{3cm}} & Корытов П.В.\\
        Преподаватель & & \underline{\hspace{3cm}} & Преподаватель П.П. 
    \end{tabularx}

    \vspace{1cm}
    Санкт-Петербург \\
    \the\year{}
\end{titlepage}
\newpage
\tableofcontents{}
\newpage

\section{Ход работы}
\subsection{Настройка среды}
Использованное ПО
\begin{enumerate}
    \item \textbf{NeoVim} --- редактор\\
    Для разработки использованы плагины:
    \begin{itemize}
        \item \texttt{octol/vim-cpp-enhanced-highlight} --- подсветка синтаксиса C++11, C++14, C++17
        \item \texttt{zchee/deoplete-clang} --- подключает clang к движку асинхронной проверки синтаксиса deoplete
        \item \texttt{derekwyatt/vim-fswitch} --- переключение между \texttt{.hpp} и \texttt{.cpp}\\
    \end{itemize}

    Указаны пути до clang:
    \begin{lstlisting}
let g:deoplete#sources#clang#libclang_path = '/usr/lib/llvm-6.0/lib/libclang.so.1'
let g:deoplete#sources#clang#clang_header = '/usr/lib/llvm-6.0/lib/clang/6.0.0/include'
    \end{lstlisting}

    \item \XeLaTeX{} --- написание и сборка отчёта

    \item \textbf{clang} --- проверка синтаксиса C/C++\\
    Установка --- \texttt{sudo apt install clang}\\
    Указаны опции запуска \texttt{-Wall -std=c++17}

    \item \textbf{CMake} --- система сборки\\
    Установка --- \texttt{sudo apt install cmake}

    \item \textbf{Google Test} --- фрейморк юнит-тестирования C++\\
    Нужно загрузить и собрать пакет:
    \begin{lstlisting}
sudo apt install libclang-dev
cd /usr/src/googletest/googletest
sudo cmake .
sudo make .
sudo ln *.a /usr/lib
    \end{lstlisting}
\end{enumerate}

\end{document}
